\documentclass[11pt]{article}   	% use "amsart" instead of "article" for AMSLaTeX format
\usepackage{geometry}                		% See geometry.pdf to learn the layout options. There are lots.
\geometry{letterpaper}                   		% ... or a4paper or a5paper or ... 
%\geometry{landscape}                		% Activate for rotated page geometry
%\usepackage[parfill]{parskip}    		% Activate to begin paragraphs with an empty line rather than an indent
\usepackage{graphicx}				% Use pdf, png, jpg, or eps§ with pdflatex; use eps in DVI mode
								% TeX will automatically convert eps --> pdf in pdflatex		
\usepackage{amssymb}
\usepackage{amsmath}
\usepackage{indentfirst}
%SetFonts

%SetFonts
\setlength{\parskip}{1em}




\begin{document}
\title{Individual Development Plan - Student Report}
\author{Student Annual Self Evaluation}

\maketitle

\section{Progress over the last year 2017-18}%
Quantum chemistry is successful in treating ground state and low-lying excited states of molecules. Remaining scientific challenges in the field include treatment of relativistic effects, multireference systems, and more general excited states. At the intersection of electronic states, standard coupled-cluster (CC) methods fail due to the non-hermitian nature of it's transformed Hamiltonian ($\bar{H}$). In order to solve this problem we are developing a unitary version of coupled cluster theory (UCC).

I have been developing a python based Automatic Expression Generator (AEG) code to carry out derivation of working equations for UCC. This is to remove a major bottleneck, i.e, the derivation of working equations is too complicated to be done by hand. My AEG computes products of second quantised operators based on Wick's theorem.

The present form of the AEG code can handle the following tasks : \\
1. Do Wick's Theorem type contraction and calculation of commutator of any two strings of creation and annihilation operators. \\
2. Collect terms from the above contraction and compare them (add/subtract) to form simple expressions and output operator expressions.\\
3. Do Nested commutator calculations such as $[[V,T1],T1]$.\\
4. Select specific $\bar{H}$ terms such as $H_{a,i}, H_{i,j}$ etc.  

With the AEG code, I have verified the energy and amplitude equations of the UCC3 method and have co-authored the UCC3 manuscript (Unitary Coupled Cluster based self consistent polarisation propagator theory: A third order formulation and pilot applications, J. Liu, A. Asthana, L. Cheng, and D. Mukherjee, submitted to the Journal of Chemical Physics (2018)).

I undertook a course on parallel programming from Computer Science Department at JHU in my first year and learnt skills to parallelize algorithms and make use of supercomputing facilities. I contributed to make an OpenMP based implementation to parallelize a FORTRAN code of relativistic coupled cluster theory and co-authored the paper (Two-component relativistic coupled-cluster methods using mean-field spin-orbit integrals, J. Liu, Y. Shen, A. Asthana, and L. Cheng, The Journal of Chemical Physics, 148 (3), 034106 (2018)).

Apart from these projects, I have participated in a summer school organised at Virginia Tech in June 2017. This was a 4 day school taught by experts in the area of molecular response properties where I worked on problem sheets and participated in classroom discussions with peers and teachers. I learnt about quantum mechanical theories to study the response of molecules when probed by electromagnetic fields.

I was involved in two TA duties this year. TA in 'Introduction to Computation Chemistry' in the Fall 2017 and Lab TA in 'Intro Chem' in the Spring Semester 2017-18.
\section{Plans and requirements over the next year}
Due to the presence of both excitation and de-excitation operators in UCC formalism, the Baker-Campbell-Hausdorff expansion form of $\bar{H}$ is non-terminating. The problem lies in finding a suitable truncation scheme for the $\bar{H}$. We have adopted an expansion scheme using Bernoulli numbers as expansion coefficients. We would like to explore possible truncation schemes based on truncation to a certain rank of commutator with respect to excitation operators. 

To deal with third order and higher order commutator in UCC, we need more optimised version of AEG program. I plan to redesign my current code to optimise  algorithms for better performance and implement universal definitions of object classes. I plan to complete this task by the end of Fall 2018. 

I will also be working on a computational project in collaboration with Bragg group in the calculation of polarisation anisotropy for the molecule 9,10-Bis-(phenyl-ethynyl)anthracene (BPEA). The project involves accurate computation of excited states.

I plan to give my department oral and GBO in the fall semester of 2018. I have started preparations for a presentation of my current and future research work and also to work on my overall understanding of Chemical Theory and Computation research. 
\end{document}  
