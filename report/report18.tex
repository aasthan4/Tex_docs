\documentclass[11pt]{article}   	% use "amsart" instead of "article" for AMSLaTeX format
\usepackage{geometry}                		% See geometry.pdf to learn the layout options. There are lots.
\geometry{letterpaper}                   		% ... or a4paper or a5paper or ... 
%\geometry{landscape}                		% Activate for rotated page geometry
%\usepackage[parfill]{parskip}    		% Activate to begin paragraphs with an empty line rather than an indent
\usepackage{graphicx}				% Use pdf, png, jpg, or eps§ with pdflatex; use eps in DVI mode
								% TeX will automatically convert eps --> pdf in pdflatex		
\usepackage{amssymb}
\usepackage{amsmath}
\usepackage{indentfirst}
%SetFonts

%SetFonts
\setlength{\parskip}{1em}




\begin{document}
\title{Annual Report 2018}
\author{Ayush Asthana : Cheng Lab}
\date{March 31 2018}							% Activate to display a given date or no date

\maketitle

\section{Progress over the last year 2017-18}%
Quantum chemistry is successful in treating ground state and low lying excited states of molecules. Remaining scientific challenges in the field include treatment of relativistic effects and multireference character, and description of more general excited states. At the intersection of electronic states, standard coupled cluster methods produce inconsistencies due to the non-hermitian nature of transformed Hamiltonian ($\bar{H}$) in Coupled Cluster (CC). As a suitable alternation we are developing a unitary version of coupled cluster theory where the $\bar{H}$ is hermitian ensuring a real energy value.\\ 

In the development of Unitary Coupled Cluster theory (UCC) a major bottle neck is the derivation of working expressions for the calculation of energy and wavefunction. These expressions are written in second quantised operators (creation and annihilation operators) and Wick's theorem is used for taking a product of strings of such operators. Diagrams derived from Wick's theorem are used to hand-derive the working equations for electronic structure methods. Due to the nature of $\bar{H}$ in UCC formalism, the number of diagrams for working equations become very large (of order of hundreds). Hand-deriving so many diagrams can be time-taking and error prone. This demands for an automatic expression generation tool that can handle the derivation process.\\

To carry out the derivation process through Wick's Theorem and generate the expressions automatically, I have been developing a python based program. The present form of the Automatic Expression Generator (AEG) code can handle the following tasks : \\
1. Do Wick's Theorem type contraction and calculation of commutator of any two strings of creation and annihilation operators. \\
2. Collect terms from the above contraction and compare them (add/subtract) to form simple expressions and output operator expressions.\\
3. Do Nested commutator calculations such as $[[V,T1],T1]$.\\
4. Select specific $\bar{H}$ terms such as $H_{a,i}, H_{i,j}$ etc.  \\

I contributed to successfully verify the energy and amplitude terms of the UCC3 method through the AEG code and co-authored the following manuscript submitted for review - Unitary Coupled Cluster based self consistent polarisation propagator theory: A third order formulation and pilot applications - J. Liu, A. Asthana, L. Cheng and D. Mukherjee submitted to the Journal of Chemical Physics 2018\\

I undertook a course on parallel programming from Computer Science Department at JHU in my first year and learnt skills to parallelize algorithms and make use of supercomputing facilities. I contributed to make an OpenMP based implementation to parallelize a FORTRAN code of relativistic coupled cluster theory and co-authored the following paper published this year - Two-component relativistic coupled-cluster methods using mean-field spin-orbit integrals
J. Liu, Y. Shen, A. Asthana, L. Cheng - The Journal of Chemical Physics, 148 (3), 034106 (2018)\\

Apart from these projects, I have participated in a summer school organised at Virginia Tech in June 2017. I learnt about quantum mechanical theories to study the response of molecules when probed by electromagnetic fields. This was a 4 day school tought by experts in the area of molecular response properties where I worked on problem sheets and participated in classroom discussions with peers and teachers.\\

I was involved in two TA duties this year. TA in 'Introduction to Computation Chemistry' in the Fall 2017 and Lab TA in 'Intro Chem' in the Spring Semester 2017-18.\\
\section{Plans and requirements over the next year}
Due to the presence of both excitation and de-excitation operators in UCC formalism, the Baker-Campell-Housdoff expansion form of $\bar{H}$ is non terminating. So the problem lies in finding a suitable truncation scheme for the $\bar{H}$. We have adopted an expansion scheme using Bernoulli numbers as expansion coefficients and separated the terms based on the degree of commutator with excitation operators. At present we have implemented second order commutators and would like to calculate the third order commutators as a next step to explore suitable truncation possibility. Third order commutators are expected to generate many hundreds of terms with added computational demands. 

To meet the needs for higher order commutator in our expansion scheme of UCC, we need more optimised version of AEG program. I plan to redesign my current code to optimise  algorithms for better performance and implement universal definitions of object classes. This task is expected to take part of my time till the end of the fall semester of 2018. 

I will also be working on a computational project in collaboration with Bragg group in the calculation of polarisation anisotropy for the molecule 9,10-Bis-(phenyl-ethynyl)anthracene (BPEA). The calculations involve accurate description and characterisation of excited states involved.

I plan to give my department oral and GBO in the fall semester of 2018. Regarding that I have started preparations to give a story to my current and future research work and also to prepare for an overall understanding of Chemical Theory and Computation research. 
\end{document}  
