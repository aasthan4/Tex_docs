\documentclass[11pt]{article}   	% use "amsart" instead of "article" for AMSLaTeX format
\usepackage{geometry}                		% See geometry.pdf to learn the layout options. There are lots.
\geometry{letterpaper}                   		% ... or a4paper or a5paper or ... 
%\geometry{landscape}                		% Activate for rotated page geometry
%\usepackage[parfill]{parskip}    		% Activate to begin paragraphs with an empty line rather than an indent
\usepackage{graphicx}				% Use pdf, png, jpg, or eps§ with pdflatex; use eps in DVI mode
								% TeX will automatically convert eps --> pdf in pdflatex		
\usepackage{amssymb}
\usepackage{amsmath}
\usepackage{indentfirst}
%SetFonts

%SetFonts
\setlength{\parskip}{1em}




\begin{document}
\title{$H_2$ Dissociation Discussions}
\author{Cheng Lab}
\date{June 13 2018}							% Activate to display a given date or no date
\maketitle
\section{Introduction to Dissociation problem}
	%write at the end
\section{Restricted Hartree Fock Solution}
	%define basis and form MOs
	A simple qualitative description of dissociation process can be made by minimal basis $H_2$ model using 1S orbitals on each H atom as our stomic basis.
	i.e $\phi_A$ and $\psi_B$ as atomic orbitals on hydrogen atom A and B. Let us write $\alpha$ and $\beta$ spin as $\phi$ and $\bar{\phi}$ respectively. \\
	Restricted Hartree Fock orbitals, determined by pure symmetry considerations are :\\ 
	$\phi_{\sigma}=\frac{1}{\sqrt{1+S_{12}}}(\phi_A+\phi_B)$\\	
	$\phi_{\sigma}^*=\frac{1}{\sqrt{1+S_{12}}}(\phi_A+\phi_B)$\\
	The determinant then is\\
	$|\phi_{\sigma}\bar{\phi}_{\sigma}>=\frac{1}{\sqrt{2}}matrix$\\
	$|\phi_{\sigma}\bar{\phi}_{\sigma}>=\frac{1}{\sqrt{2}}matrix$\\
	The energy of this determinant based on applying the Fock operator will be $2h_{\sigma \sigma} + J_{\sigma \sigma}$ (one columbic interaction 
	and no exchange possible between electrons of opposite spin) where h is the core 1 electron energy (kinetic and nuclear potential energy)
	and J is the columbing 2 electron integral. Let us write our ground state determinant in terms of atomi orbitals.\\
	\begin{equation}
	\begin{split}
		|\phi_{\sigma}\bar{\phi}_{\sigma}> &=\frac{1}{\sqrt{1+S_{12}}}|(\phi_A+\phi_B)(\bar(\phi)_A+\bar{\phi}_B)>\\
					  &=\frac{1}{\sqrt{1+S_{12}}}(|\phi_A \phi_A>+|\phi_A \phi_B>+|\phi_B \phi_A+|\phi_B \phi_B>)\\
	\end{split}
	\end{equation}
	Note that two out of four configurations of RHF ground state determinant are show ionic bond character. These determinants are the reason of
	problamatic behavious as we shall see later. \\
	\begin{equation}
	derive energy based on spaial orbitals and then atomic orbitals. 
	\end{equation}
	At the dissociation limit, $H_2$ molecule will be represented as two $H$ atoms at infinite distace so they do not interact with each other. 
	Thus in a correct description of dissociation limit, energy of dissoiated $H_2$ molecule should go to two times energy of $H$ atom i.e 
	$2 h_A$. The energy calculated by RHF is an overestimation by $J_{\sigma \sigma}$. 
	Unfortunately, $J_{\sigma \sigma}$ does not go to zero as shown below. \\
	\begin{equation}
	\begin{split}
		J_{\sigma \sigma} &=\int_r \phi_{\sigma}^*(1) \phi_{\sigma}(1)r_{12}\phi_{\sigma}^*(2) \phi_{\sigma}(2)\\
			&=\int_r \phi_A^*(1) \phi_A(1)r_{12}\phi_A^*(2) \phi_A(2)+\int_r \phi_B^*(1) \phi_B(1)r_{12}\phi_B^*(2) \phi_B(2)\\
			&\neq 0
	\end{split}
	\end{equation}
	\subsection{Discussion of results}
	%discuss the source of the problem
	If we look at equation x carefully we find that there are four determinants contributing to the molecular configuration, two of which are ionic in nature (1 and 4th). 
	These determinants do not contribute to the correct dissociated $H_2$ atoms as the atomic orbitals are spacially separated and ionic type dissociation cannot occur. 
	RHF molecular orbitals are unable to dissociate a molecule as they have same spacial $\Psi$ for both spin $\alpha$ and $\beta$. Thus we need additional degree of fredom in our orbital wavefunction to include different spacial wavefunctions for spin states. An Unrestricted 
	Hartree Fock (UHF) can solve this problem. \\
\section{Unrestricted Hartree Fock Solution}
	Restricted set of orbitals were generated purely by symmetry. Unrestricted molecualar orbitals have different 
	spacial functions for $\alpha$ and $\beta$ spin of the same orbital. A formulation to inorporate additional degree of freedom in symmetry 
	determined RHF molecular orbitals is to form UHF orbitals as a linear combination of RHF orbitals. \\
	%We know that in the dissociated state $H_2$ minimal basis configuration will be $|\phi_A \bar{\phi}_B>$ where $\phi_a$ and $\phi_B$ are atomic orbitals of $H$ atoms. We can thus relax our RHF orbitals in such a way that they 
	%finally form the above configuration at the dissociation limit. \\
	%equation for parameter addition \\
	\begin{equation}

		\phi_{\sigma}^{\alpha}=cos\theta \phi_{\sigma}+sin\theta \phi_{\sigma^*}\\
		\phi_{\sigma}^{\beta}=cos\theta \phi_{\sigma}-sin\theta \phi_{\sigma^*}\\
	\end{equation}
	\begin{equation}
		\phi_{\sigma^*}^{\alpha}=cos\theta \phi_{\sigma^*}+sin\theta \phi_{\sigma}\\
		\phi_{\sigma^*}^{\beta}=cos\theta \phi_{\sigma^*}-sin\theta \phi_{\sigma}\\
	\end{equation}
	The added degree of freedome here is in $\theta$ which can be varied from 0 to $45^o$. For our perposes it is sufficient to consider the 
	values at the 0 and $45^o$. At $0^o$ :\\

	\begin{equation}
	\begin{split}
		\phi_{\sigma}^{\alpha}&=\phi_{\sigma}
		\phi_{\sigma}^{\beta}&=\phi_{\sigma}
		\phi_{\sigma^*}^{\alpha}&=\phi_{\sigma^*}
		\phi_{\sigma^*}^{\beta}&=\phi{\sigma^*}
	\end{split}
	\end{equation}
	These are RHF orbitals as discussed in the above section. At $\theta=45^o$ we have :\\
	\begin{equation}
	\begin{split}
		\phi_{\sigma}^{\alpha}&=\frac{1}{\sqrt{2}}(\phi_{\sigma}+\phi_{\sigma^*})
		\phi_{\sigma}^{\beta}&=\frac{1}{\sqrt{2}}(\phi_{\sigma}-\phi_{\sigma^*})
		\phi_{\sigma*}^{\alpha}&=\frac{1}{\sqrt{2}}(\phi_{\sigma^*}+\phi_{\sigma})
		\phi_{\sigma*}^{\beta}&=\frac{1}{\sqrt{2}}(\phi_{\sigma^*}-\phi_{\sigma})
	\end{split}
	\end{equation}
	In $\theta=45^o$ case, taking the dissociation limit case $\phi_{\sigma}^{\alpha}\neq \phi_{\sigma}^{\beta}$ and more specifially
	$\phi_{\sigma}^{\alpha}=\phi_A$ and $\phi_{\sigma}^{\beta}=\phi_B$. Thus the determinant $|\phi_{\sigma}^{\alpha}\phi_{\sigma}^{\beta}>
	=|\phi_A\bar{\phi}_B>$. These are the UHF orbitals.\\ 
	Considering the dissociated case of $H_2$ molecule, now we calculate the energy of the ground state determinant at the dissociation limit. \\
	\begin{equation}
		\begin{split}
			lim_{r->\inf}<\Phi_o|H|\Phi_o>&=<\phi_{\sigma}^{\alpha}\phi_{\sigma}^{\alpha}|H|\phi_{\sigma}^{\alpha}\phi_{\sigma}^{\alpha}>\\
				&=<\phi_A\phi_B|H|\phi_A\phi_B>\\
				&=\frac{1}{2}(h_AA+h_BB)\\
		\end{split}
	\end{equation}
	As shown here unlike RHF, UHF orbitals show correct behaviour of energy at the dissociation limit of $H_2$ molecule.\\
	\subsection{Discussion}
	We see that the energy by UHF wavefunction $lim_{r->\inf}|\Phi_o>=|\phi_{\sigma}^{\alpha}\phi_{\sigma}^{\beta}>$ goes to
	correct limit but the wavefunction does not, as discussed in the following sections. \\
	\section{Two Orbital two electron system}
	In order to understand what the correct behavior of the wavefunction at the dissociation limit be, let us consider a two electron in two 
	orbital $\phi_1 \phi_2$ case, just like our $H_2$ system. Two electrons can be arranged in two orbitals in the following 6 ways. \\
	\begin{equation}
		|\phi_1\bar{\phi}_1>, |\phi_2\bar{\phi}_2>, |\phi_1\bar{\phi}_2>, |\bar{\phi}_1\phi_2>, |\phi_1\phi_2>, |\phi_2\phi_1>\\ 
	\end{equation}
	In the above determinants, 3rd and 4th determinant are not pure spin state. \\
	$$S^2|\phi_1\bar{phi}_2>=complete$$
	By taking a linear combination of 3rd and 4th terms, we get :\\
	\begin{equation}
		
	\end{equation}
	\subsection{Spin Operator}
	Spin Operator is a vector quantity defined as follows :
	$\vv{s}=s_x \vv{i}+s_y\vv{j} + s_z\vv{k}$\\
	The x, y and z components are defined as observed experimentally through the Glen Fredlich experiment. 
	$s_x=$\\
	$s_y=$\\
	$s_z=$\\
	A more convenient notation used is to define ladder operators
	$s_+=s_x+is_y$\\
	$s_-=s_x-is_y$\\
	The action of these ladder operators can be seen as taking the spin state up the ladder through $s_+$ and down the ladder through $s_-$.\\
	The x,y and z components of the spin operator do not commute with each other.\\
	$\[s_x,s_y\]=i\h$
	$\[s_x,s_y\]=i\h$
	$\[s_x,s_y\]=i\h$
	The action of these operators can be seen on spin states defined by $\alpha$ or (1,0) in matrix notation  and $\beta$ or (0,1).	
	Their operators can also be defined through Pauli matrices.
	$s_x=\matrix{}{}$
	$s_x=\matrix{}{}$
	$s_x=\matrix{}{}$
	$s_+$
	$s_-$
	The action of these matrices on the vector form of spin state gives the resulting spin state vector. Such as :
	$s_x |\alpha>=()()=$
	Through the above action we can come up with the following relations. 
	$s_+|\alpha>=0  s_-|\alpha>=|\beta>$
	$s_+|\beta>=|\alpha>  s_-|\beta>=0$
	The above definitions are valid for atomic orbitals. The following set of derivations  can be made for molecular orbitals. 
	S
	
	Since $S^2$ commutes with $H$, exact molecular orbitals are also eigenfunctions of $S^2$ operators and called pure spin states. Since we use an approximate solution
	to wave function, our wavefunction $\Psi$ can be at times not a pure spin state. In such cases we can make the wavefunction pure spin states by taking a linear combination
	of wavefunctions produced. 
	\subsection{Spin Adapted Configurations}
	Let us consider the example of $H_2$ to illustrate spin adapted configurations. The ground state of $H_2$ will have $\alpha$ and $\beta$ electrons in its lowest energy 
	molecular orbital. 
\section{Configurational Interaction Solution}
	One way to look at our RHF solution of $H_2$ dissociation is to follow the two molecular orbitals produced $\phi_{\sigma}$ and $\phi_{sigma^*}$. 
	The energy of configuration $|\phi_{\sigma} \bar{\phi}_{\sigma}$ is $2h_{\sigma} +J_{\sigma \sigma}$ and of configuration $|\phi_{\sigma^*} \bar{\phi}_{\sigma^*}$ is $2h_{\sigma^*} +J_{\sigma^* \sigma^*}$. If we expand the molecular orbitals and write the energy in terms of atomic orbitals, \\
	$2h_{\sigma} +J_{\sigma \sigma}=h_AA+h_AB+h_BA+h_BB+J_AA+J_BB$\\
	$2h_{\sigma} +J_{\sigma \sigma}=h_AA-h_AB-h_BA+h_BB_J_AA+J_BB$\\
	At the equilibrium distance, |\phi_{\sigma}> is clearly lower in energy but at the dissociation limit, $lim_{r->\inf} h_AB=h_BA=0$. Thus the two states
	become degenerate. HF only calculating the energy of one of these states i.e $\phi_{\sigma}$. The energy is even minimised by what is known as the non-
	dynamic correlation. One method to deal with this psudo degeneracy is by taking a linear combination of the two states and using Configuration Interaction 
	to solve at the dissociated state. \\
	$|\Phi>=c1 |\phi_{\sigma}\bar{\phi}_{\sigma}>+c2|\phi_{\sigma^*}\bar{\phi}_{\sigma^*}>$\\
	\matrix{}{}{} equation\\
	where the Hamiltonian coupling terms are :
	$H12$ all \\
	$E=c1H1$
	We have an energy equation constrained by the normalisation equation $<\Phi|\Phi>=1$ which gives us $c1^2+c2^2=1$.
	To solve for the minimum in energy based on the above constraint, we form the Lagrange function and find the minimum. 
	$L=$
	The above equation gives us 4 equations.
	\subsection{Discussion}
	\subsection{Secular Equations and its solution}
\end{document}  
