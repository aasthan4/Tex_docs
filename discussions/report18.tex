\documentclass[11pt]{article}   	% use "amsart" instead of "article" for AMSLaTeX format
\usepackage{geometry}                		% See geometry.pdf to learn the layout options. There are lots.
\geometry{letterpaper}                   		% ... or a4paper or a5paper or ... 
%\geometry{landscape}                		% Activate for rotated page geometry
%\usepackage[parfill]{parskip}    		% Activate to begin paragraphs with an empty line rather than an indent
\usepackage{graphicx}				% Use pdf, png, jpg, or eps§ with pdflatex; use eps in DVI mode
								% TeX will automatically convert eps --> pdf in pdflatex		
\usepackage{amssymb}
\usepackage{amsmath}
\usepackage{mathtools}
\usepackage{indentfirst}
%SetFonts

%SetFonts
\setlength{\parskip}{1em}
\begin{document}
\title{$H_2$ Dissociation Discussions}
\author{Cheng Lab}
\date{June 13 2018}							% Activate to display a given date or no date
\maketitle
\section{Introduction to Dissociation problem}
	%write at the end
\section{Restricted Hartree Fock Solution}
	%define basis and form MOs
	A simple qualitative description of dissociation process can be made by minimal basis $H_2$ model using 1S orbitals on each H atom as our stomic basis.
	i.e $\phi_A$ and $\psi_B$ as atomic orbitals on hydrogen atom A and B. Let us write $\alpha$ and $\beta$ spin as $\phi$ and $\bar{\phi}$ respectively. \\
	Restricted Hartree Fock orbitals, determined by pure symmetry considerations are :\\ 
	\begin{equation}
	\begin{split}
	\phi_{\sigma}=\frac{1}{\sqrt{2(1+S_{AB})}}(\phi_A+\phi_B) \\	
	\phi_{\sigma^*}=\frac{1}{\sqrt{2(1-S_{AB})}}(\phi_A-\phi_B) \\
	\end{split}
	\end{equation}
	where $S_{AB}$ is the overlap of atomic orbitals $\phi_A$ and $\phi_B$. The ground state staler determinant can be written as :\\
	\begin{equation}
	\begin{split}
	|\phi_{\sigma}\bar{\phi}_{\sigma}>=\frac{1}{\sqrt{2}} \begin{vmatrix}\phi_{\sigma}(1) 
	& \bar{\phi}_{\sigma}(1) \\  \phi_{\sigma}(2) & \bar{\phi}_{\sigma}(2) \\ \end{vmatrix}  \\
	\end{split}
	\end{equation}
	%edit the language
	
	The energy of this determinant based on applying the Fock operator will be $2h_{\sigma \sigma} + J_{\sigma \sigma}$ (one columbic interaction 
	and no exchange possible between electrons of opposite spin) where h is the core 1 electron energy (kinetic and nuclear potential energy)
	and J is the columbing 2 electron integral. Let us write our ground state determinant in terms of atomi orbitals.\\
	\begin{equation}
	\begin{split}
		|\phi_{\sigma}\bar{\phi}_{\sigma}> &=\frac{1}{\sqrt{2}}\big[ \phi_{\sigma}(1)\bar{\phi}_{\sigma}(2)
				-\phi_{\sigma}(2)\bar{\phi}_{\sigma}(1)\big]\\
			&=\frac{1}{2\sqrt{2}(1+S_{AB})}\big[(\phi_A(1)+\phi_B(1))(\bar{\phi}_A(2)+\bar{\phi}
			_B(2))-(\phi_A(2)+\phi_B(2))(\bar{\phi}_A(1)+\bar{\phi}_B(1))\big]\\
	\end{split}
	\end{equation}
	Expanding and rearranging, we get :\\
	\begin{equation}
	\begin{split}
	|\phi_{\sigma}\bar{\phi}_{\sigma}>=\frac{1}{2\sqrt{2}(1+S_{AB})}(|\phi_A \bar{\phi}_A>+|\phi_A \bar{\phi}_B>+|\phi_B \bar{\phi}_A>+|\phi_B \bar{\phi}_B>)\\
	\end{split}
	\end{equation}
	Note that two out of four configurations of RHF ground state determinant show ionic bond character. These determinants are the reason of
	problamatic behavious in dissociation of $H_2$ as we shall see in the later sections. We can evaluate the total energy of this configuration by inspection. Each electron has a core kinetic and nuclear repulsion 
	energy part $h_{\sigma \sigma}$ and the two electron have a columbic repulsion energy
	 $J_{\sigma \sigma}$ :\\
	\begin{equation}
		<\phi_{\sigma}\bar{\phi}_{\sigma}|H|\phi_{\sigma}\bar{\phi}_{\sigma}>
		=2 h_{\sigma \sigma}+J_{\sigma \sigma}\\
	\end{equation}
	We will analyse the two parts of energy one by one to see their behavior at the dissociation limit. \\
	\begin{equation}
	\begin{split}
		2 h_{\sigma \sigma}&=2 (\phi_{\sigma}\bar{\phi}_{\sigma}|h|
		\phi_{\sigma}\bar{\phi}_{\sigma})\\
		&=\frac{2}{2(1+S_{AB})}((\phi_A+\phi_B)(\bar{\phi}_A+\bar{\phi}_B)|h|(\phi_A+\phi_B)(\bar{\phi}_A+\bar{\phi}_B))\\
		&=\frac{1}{1+S_{AB}}(h_{AA}+h_{AB}+h_{BA}+h_{BB})\\
	\end{split}
	\end{equation}
	As $R\rightarrow \infty$ $S_{AB}$ and $h_{AB}\rightarrow 0$. Thus $2h_{\sigma \sigma}=h_{AA}+h_{BB}$ at the dissociation
	limit. Using eq(4) and expanding $J_{\sigma \sigma}$ in atomic orbitals at dissociation limit, we get :
	\begin{equation}
	\begin{split}
		J_{\sigma \sigma} &=(\phi_{\sigma} \phi_{\sigma}|\phi_{\sigma} \phi_{\sigma})\\
			&= lim_{r\rightarrow\infty}\big( (\phi_A\phi_A|+(\phi_A\phi_B|+(\phi_B\phi_A|
			+(\phi_B\phi_B|\big)\big|\\&    \big(|\phi_A\phi_A)+|\phi_A\phi_B)+|\phi_B\phi_A)+
			|\phi_B\phi_B)\big)\\
			&=(\phi_A\phi_A|\phi_A\phi_A)+(\phi_B\phi_B|\phi_B\phi_B)\\
			&\neq 0
	\end{split}
	\end{equation}
	\subsection{Discussion of results}
	%discuss the source of the problem
	From eq(6) and eq(7), we infer that the RHF energy of dissociated $H_2$ molecule is $h_{AA}+h_{BB}+
	J_{AA}+J_{BB}$. Dissociated $H_2$ molecule essentially means $ 2 H $atoms at $\infty$ distance so 
	they cannot interact. By inspection the energy of this system should be twice energy of $H$ atom. 
	Thus our RHF calculation is an overestimation of the exact energy.  \\
	If we look at eq(4) we find that there are four determinants contributing to the molecular configuration, two of which are ionic in nature (1st and 4th). 
	These determinants should not contribute to get the correct dissociated $H_2$ atoms as the atomic orbitals are spacially separated.
	RHF molecular orbitals are unable to dissociate a molecule as they have same spacial $\Psi$ 
	for both spin $\alpha$ and $\beta$. Thus we need additional degree of fredom in our 
	orbital wavefunction to include different spacial wavefunctions for spin states. An Unrestricted 
	Hartree Fock (UHF) can solve this problem. \\
\section{Unrestricted Hartree Fock Solution}
	Restricted set of orbitals were generated purely by symmetry. Unrestricted molecualar orbitals have different 
	spacial functions for $\alpha$ and $\beta$ spin of the same orbital. A formulation to inorporate additional degree of freedom in symmetry 
	determined RHF molecular orbitals is to form UHF orbitals as a linear combination of RHF orbitals. \\
	%We know that in the dissociated state $H_2$ minimal basis configuration will be $|\phi_A \bar{\phi}_B>$ where $\phi_a$ and $\phi_B$ are atomic orbitals of $H$ atoms. We can thus relax our RHF orbitals in such a way that they 
	%finally form the above configuration at the dissociation limit. \\
	%equation for parameter addition \\
	\begin{equation}
	\begin{split}
		\phi_{\sigma}^{\alpha}=cos\theta \phi_{\sigma}+sin\theta \phi_{\sigma^*}\\
		\phi_{\sigma}^{\beta}=cos\theta \phi_{\sigma}-sin\theta \phi_{\sigma^*}\\
	\end{split}
	\end{equation}
	\begin{equation}
	\begin{split}
		\phi_{\sigma^*}^{\alpha}=cos\theta \phi_{\sigma^*}+sin\theta \phi_{\sigma}\\
		\phi_{\sigma^*}^{\beta}=cos\theta \phi_{\sigma^*}-sin\theta \phi_{\sigma}\\
	\end{split}
	\end{equation}
	The added degree of freedome here is in $\theta$ which can be varied from 0 to $45^o$. For our perposes it is sufficient to consider the 
	values at the 0 and $45^o$. As we shall see, $\theta=0^o$ gives us RHF orbitals while $\theta\neq0^o$ 
	produces UHF orbitals. At $0^o$ :\\
	\begin{equation}
		\begin{split}
		\phi_{\sigma}^{\alpha}&=\phi_{\sigma}\\
		\phi_{\sigma}^{\beta}&=\phi_{\sigma}\\
		\end{split}
	\end{equation}
	These are RHF ground state spin orbitals as discussed earlier. At $\theta=45^o$ we have :\\
	\begin{equation}
	\begin{split}
		\phi_{\sigma}^{\alpha}&=\frac{1}{\sqrt{2}}(\phi_{\sigma}+\phi_{\sigma^*})\\
		\phi_{\sigma}^{\beta}&=\frac{1}{\sqrt{2}}(\phi_{\sigma}-\phi_{\sigma^*})\\
	\end{split}
	\end{equation}
	In $\theta=45^o$, taking the dissociation limit case we have 
	$\phi_{\sigma}^{\alpha}=\phi_A$ and $\phi_{\sigma}^{\beta}=\phi_B$ which are unrestricted set of 
	molecular orbitals.
	With the new set of orbitals our determinant becomes $|\phi_{\sigma}^{\alpha}\phi_{\sigma}^{\beta}>
	=|\phi_A\bar{\phi}_B>$.\\ 
	Now let us calculate the energy of the ground state determinant at the dissociation limit with our new UHF orbitals. \\
	\begin{equation}
		\begin{split}
			lim_{r\rightarrow \infty}<\Psi_o|H|\Psi_o>&=<\phi_{\sigma}^{\alpha}
			\phi_{\sigma}^{\beta}|H|\phi_{\sigma}^{\alpha}\phi_{\sigma}^{\beta}>\\
				&=<\phi_A\phi_B|H|\phi_A\phi_B>\\
				&=h_{AA}+h_{BB}\\
		\end{split}
	\end{equation}
	As shown here unlike RHF, UHF orbitals show correct behaviour of energy at the dissociation 
	limit of $H_2$ molecule.\\
	\subsection{Discussion}
	We see that the energy by UHF wavefunction $lim_{r\rightarrow\infty}|\Phi_o>=|\phi_{\sigma}^{\alpha}\phi_{\sigma}^{\beta}>$ goes to
	correct limit but the wavefunction is incorrect. The UHF wavefunction at the dissociation limit
	 becomes $|\phi_A\bar{\phi}_B>$. But this is not a pure spin state in the case where electrons occupy
	 different spacial orbitals, as will be discussed in the 
	later sections. The correct wavefunction is :\\
	\begin{equation}
	\begin{split}
		Lim_{r\rightarrow\infty}|\Psi_o>=\frac{1}{\sqrt{2}}(|\phi_A\bar{\phi}_B>+|\bar{\phi}_A\phi_B>)
	\end{split}
	\end{equation}

\subsection{Spin Operator}
	Spin Operator is a vector quantity defined as follows :
	\begin{equation}
		\vec{s}=s_x \vec{i}+s_y\vec{j} + s_z\vec{k}\\
	\end{equation}
	These spin operator components do not commute with each other and satisfy the following relations :
	\begin{equation}
	\begin{split}
		[s_x,s_y]=is_z,   
		[s_y,s_z]=is_x,   
		[s_z,s_x]=is_y
	\end{split}
	\end{equation}
	We can derive a set of states of the spin of a particle as eigenfunctions of $s^2$ or one of the components of $\vec{s}$ conventionally taken to be $s_z$. \\
	\begin{equation}
	\begin{split}	
		s^2|\phi>&=s(s+1)|\phi>\\
		s_z|\phi>&=m_s|\phi>
	\end{split}
	\end{equation}
	Where $s$ is the total spin of the particle and $m_s$ is the quantum number for the z conponent. 
	Since electron is a spin half system, where $m_s=\frac{1}{2} , -\frac{1}{2}$. 
	For our convinience, we can define the two states in the matrix representation as follows :\\
	\begin{equation}
		|\alpha>=\Big(\begin{matrix} 1\\ 0\end{matrix}\Big), 
		|\beta>=\Big(\begin{matrix} 0\\ 1\end{matrix}\Big)\\
	\end{equation}
	This leads us to a more convinient defition of ladder operators:
	\begin{equation}
	\begin{split}
	s_+=\bigg(\begin{matrix} 0 & 1\\ 0 & 0\end{matrix}\bigg)\\
	s_-=\bigg(\begin{matrix} 0 & 0\\ 1 & 0\end{matrix}\bigg)\\
	\end{split}
	\end{equation}
	where the action of these ladder operators can be seen as taking the spin state up 
	like 'step up' through $s_+$ and down like 'step down' through $s_-$. They are 
	defined in terms for x and y components of $\vec{s}$ as follows :
	\begin{equation}
	\begin{split}
		s_+&=s_x+is_y\\
		s_-&=s_x-is_y\\
	\end{split}
	\end{equation}
	Using eq(18), (20) and (21) we can write the x, y and z components of our spin operator in the matrix 
	representation as $s_k=\frac{1}{2} \sigma_k$. Here k=x, y and z and $\sigma_k$ is defined as follows:\\
	\begin{equation}
		\begin{split}
		\sigma_1=\bigg(\begin{matrix}0 & 1\\ 1 & 0\end{matrix}\bigg),
		\sigma_2=\bigg(\begin{matrix}0 & -i\\ i & 0\end{matrix}\bigg), 
		\sigma_3=\bigg(\begin{matrix}1 & 0\\ 0 & -1\end{matrix}\bigg)\\ 
		\end{split}
	\end{equation}
	where 1, 2 and 3 refer to x, y and z components. This matrix formalism for spin $\frac{1}{2}$ systems 
	was introduced by W. Pauli in 1926. We can also derive an expression for $s^2$ in terms of components
	of $\vec{s}$ using eq(15) and (16) :\\
	\begin{equation}
	\begin{split}
		s^2=s_+s_--s_z+s_z^2\\
		s^2=s_-s_++s_z+s_z^2\\
	\end{split}
	\end{equation}
	For manyelectron systems, the total spin sngular momentum is the sum of spin vectors for each 
	electron.\\
	\begin{equation}
	\begin{split}
		\vec{S}=\sum_{i=1}^N \vec{s}\big(i\big)\\
	\end{split}
	\end{equation}	
	Thus the total squared spin operators is :\\
	\begin{equation}
	\begin{split}
		S^2 &=\vec{S} .\vec{S}=\sum_{i=1}^N \sum_{j=1}^N \vec{s}\big(i\big)\vec{s}\big(j\big)\\
			&=S_+S_--S_z+S_z^2\\
			&=S_-S_++S_z+S_z^2
	\end{split}
	\end{equation}
	Since there is no spin part in our non-relativistic $H$, $S^2$ and $S_z$ commutes with $H$ 
	i.e $[H,S^2]=0$	and $[H,S_z]=0$. Thus the exact eigenfunctions of Hamiltonian (exact wavefunctios) 
	are also eigenfunctions of $S^2$ and $S_z$ operators.\\ 
	Approximate solutions of $H$ are not necessarily pure spin states. However we would like to convert
	them to spin adapted configurations to form eigenfunctions of $S^2$. We will describe the process 
	through an example in the following sections. \\
	

	\subsection{Two Orbital two electron system}
	To understand spin adapted configurations and the correct behavior of the wavefunction at 
	the dissociation limit be, let us consider an example of a simple case of two electrons in two 
	orbital $\phi_1 \phi_2$ case as in our minimal basis $H_2$. Two electrons can be arranged 
	in two orbitals in the following 6 ways: \\
	\begin{equation}
		|\phi_1\bar{\phi}_1>, |\phi_2\bar{\phi}_2>, |\phi_1\bar{\phi}_2>, |\bar{\phi}_1\phi_2>, |\phi_1\phi_2>, |\bar{\phi}_2\bar{\phi}_1>\\ 
	\end{equation}
	If we calculate the $S$ value for the above configurations, we can see that $|\phi_1\bar{\phi}_2>
	and  |\bar{\phi}_1\phi_2>$ are not pure spin states. \\
	\begin{equation}
	\begin{split}
	S^2|\phi_1\bar{\phi}_2>&=S^2\big[\phi_1(1)\phi_2(2)\alpha(1) \beta(2)-\phi_1(2)\phi_2(1)\alpha(2) \beta(1)\big]\\
		&=\big[\phi_1(1)\phi_2(2)-\phi_1(2)\phi_2(1)\big](\alpha(1) \beta(2)+\alpha(2)\beta(1))
	\end{split}
	\end{equation}
	We can make them eigenfunctions of $S^2$ operator by taking a linear combination of appropriate terms
	\\
	\begin{equation}
	\begin{split}
		|\prescript{1}{}{\Psi}>=\frac{1}{\sqrt{2}}(|\phi_1\bar{\phi}_2>+|\phi_1\bar{\phi}_2>)\\	
		|\prescript{3}{}{\Psi}>=\frac{1}{\sqrt{2}}(|\phi_1\bar{\phi}_2>-|\phi_1\bar{\phi}_2>)\\
	\end{split}
	\end{equation}
	As discussed here, we now have three triplet and three singlet spin adapted states 
	from two electrons in two orbitals model. Writing in terms of $H_2$ case that we discussed in 
	section (2) and (3) we have the following singlet states :\\
	\begin{equation}
	\begin{split}
		|\prescript{1}{}{\Psi_1}>&=|\phi_A\bar{\phi}_{\sigma}>\\
		|\prescript{1}{}{\Psi_2}>&=|\phi_{\sigma^*}\bar{\phi}_{\sigma^*}>\\
		|\prescript{1}{}{\Psi_3}>&=\frac{1}{\sqrt{2}}(|\phi_{\sigma}\bar{\phi}_{\sigma^*}>+
		|\phi_{\sigma}\bar{\phi}_{\sigma^*}>)\\	
	\end{split}
	\end{equation}
	Similarly the triplet states are :\\
	\begin{equation}
	\begin{split}
		|\prescript{3}{}{\Psi_4}>&=|\phi_{\sigma}\phi_{\sigma^*}>\\
		|\prescript{3}{}{\Psi_5}>&=|\bar{\phi}_{\sigma^*}\bar{\phi}_{\sigma}>\\
		|\prescript{3}{}{\Psi_6}>&=\frac{1}{\sqrt{2}}(|\phi_{\sigma}\bar{\phi}_{\sigma^*}>-
		|\phi_{\sigma}\bar{\phi}_{\sigma^*}>)\\
	\end{split}
	\end{equation}
\section{Configurational Interaction Solution}
	Let us apply configuration Internaction method to treat the $H_2$ dissociation problem. The ground state 
	$|\prescript{1}{}{\Psi_1}>=|\phi_A\bar{\phi}_{\sigma}>$ is a singlet, so we will include the three singlet configurations 
	described in the above section in our CI wavefunction.\\
	\begin{equation}
	\begin{split}
		|\Phi_o>=|\prescript{1}{}{\phi_1}>+c1|\prescript{1}{}{\phi_2}>+c2|\prescript{1}{}{\phi_3}>
	\end{split}
	\end{equation}
	As a next step, we can simplify our $\Phi_o$ by taking symmetry into consideration. Both $\Phi_o$ and the doubly excited determinant are gerade and 
	will mix while $\prescript{1}{}{\Phi_3}$ is ungerade and thus will not. So we can simplify our $\Phi_{CI}$ as :\\
	\begin{equation}
		|\Phi_o>=|\prescript{1}{}{\phi_1}>+c1|\prescript{1}{}{\phi_2}>
	\end{equation}
	One way to look at our RHF solution of $H_2$ dissociation is to follow the two molecular orbitals produced $\phi_{\sigma}$ and $\phi_{sigma^*}$. 
	The energy of configuration $|\phi_{\sigma} \bar{\phi}_{\sigma}$ is $2h_{\sigma} +J_{\sigma \sigma}$ and of configuration 
	$|\phi_{\sigma^*} \bar{\phi}_{\sigma^*}$ is $2h_{\sigma^*} +J_{\sigma^* \sigma^*}$. If we expand the molecular orbitals and 
	write the energy in terms of atomic orbitals, \\
	\begin{equation}
	2h_{\sigma} +J_{\sigma \sigma}=h_{AA}+h_{AB}+h_{BA}+h_{BB}+J_{AA}+J_{BB}\\
	2h_{\sigma} +J_{\sigma \sigma}=h_{AA}-h_{AB}-h_{BA}+h_{BB}-J_{AA}+J_{BB}\\
	\end{equation}
	At the equilibrium distance, $|\phi_{\sigma}>$ is clearly lower in energy but at the dissociation limit, $lim_{r\rightarrow\infty} h_AB=h_BA=0$. Thus the two states
	become degenerate. In HF one is calculating the energy of one of these states i.e $\phi_{\sigma}$. The energy is even minimised by what is known as the non-
	dynamic correlation. Thus taking a Configuration Interaction with both the states in $\Phi_{CI}$ gives the correct energy.\\
	$|\Phi>=c1 |\phi_{\sigma}\bar{\phi}_{\sigma}>+c2|\phi_{\sigma^*}\bar{\phi}_{\sigma^*}>$\\
	matrix equation\\
	where the Hamiltonian coupling terms are :
	$H12$ all \\
	$E=c1H1$
	We have an energy equation constrained by the normalisation equation $<\Phi|\Phi>=1$ which gives us $c1^2+c2^2=1$.
	To solve for the minimum in energy based on the above constraint, we form the Lagrange function and find the minimum. 
	$L=$
	The above equation gives us 4 equations.
	\subsection{Discussion}
	\subsection{Secular Equations and its solution}
\end{document}  
