\documentclass[11pt]{article}   	% use "amsart" instead of "article" for AMSLaTeX format
\usepackage{geometry}                		% See geometry.pdf to learn the layout options. There are lots.
\geometry{letterpaper}                   		% ... or a4paper or a5paper or ... 
%\geometry{landscape}                		% Activate for rotated page geometry
%\usepackage[parfill]{parskip}    		% Activate to begin paragraphs with an empty line rather than an indent
\usepackage{graphicx}				% Use pdf, png, jpg, or eps§ with pdflatex; use eps in DVI mode
								% TeX will automatically convert eps --> pdf in pdflatex		
\usepackage{amssymb}
\usepackage{amsmath}
\usepackage{indentfirst}
%SetFonts

%SetFonts
\setlength{\parskip}{1em}




\begin{document}
\title{$H_2$ Dissociation Discussions}
\author{Cheng Lab}
\date{June 13 2018}							% Activate to display a given date or no date
\maketitle
\section{Introduction to Dissociation problem}
	%write at the end
\section{Restricted Hartree Fock Solution}
	%define basis and form MOs
	For minimum basis $H_2$ we have two 2s orbitals, one on each H atom as our basis.
	i.e $\phi_A$ and $\psi_B$ as atomic orbitals on hydrogen atom A and B. Let $\bar{\phi_A}$ and $\bar{\phi_B}$
	denote the $\beta$ spin. \\
	Restricted Hartree Fock orbitals will be determined by pure symmetry considerations.\\ 
	$\phi_{\sigma}=\frac{1}{\sqrt{1+S_{12}}}(\phi_A+\phi_B)$\\	
	$\phi_{\sigma}=\frac{1}{\sqrt{1+S_{12}}}(\phi_A+\phi_B)$\\
	The determinant then is\\
	$|\phi_{\sigma}\bar{\phi}_{\sigma}>=\frac{1}{\sqrt{2}}matrix$\\
	$|\phi_{\sigma}\bar{\phi}_{\sigma}>=\frac{1}{\sqrt{2}}matrix$\\
	The energy of this determinant based on applying the Fock operator will be $2h_{\sigma \sigma} + J_{\sigma \sigma}$
	where h is the core 1 electron part and J is the columbing 2 electron integral. Let us evaluate the energy based on atomic orbitals.\\
	determinant ->
	To correctly describe the dissociation limit, energy of dissoiated $H_2$ molecule should go to two times energy of $H$ atom. The energy calculated by RHF is an overestimation by $J_{\sigma \sigma}$. 
	$J_{\sigma \sigma}$ should go to zero for correct description which it doesnt as shown below. \\
	$J_{\sigma \sigma}=$\\
	
	\subsection{Discussion of results}
	%discuss the source of the problem
	If we look at equation x carefully we find that there are four determinants contributing to the molecular configuration, two of which are ionic in nature (1 and 4th). 
	These determinants do not contribute to the correct dissociated $H_2$ atoms as the atomic orbitals are spacially separated. RHF molecular orbitals are unable to include this spacial separation as they 
	include same spacial $\Psi$ for both spin $\alpha$ and $\beta$. Thus we need additional degree of fredom in our orbital wavefunction to include different spacial wavefunctions for spin states. An Unrestricted 
	Hartree Fock (UHF) can solve this problem discussed in the next section. \\
\section{Unrestricted Hartree Fock Solution}
	Restricted set of orbitals were generated purely by symmetry. Unrestricted orbitals are generated through adding additional degree of freedom to make sure the spacial part of the orbitals wavefunction 
	can be relaxed. A formulation for $H_2$ minimal basis case is described below. \\
	We know that in the dissociated state $H_2$ minimal basis configuration will be $|\phi_A \bar{\phi}_B>$ where \phi_a and \phi_B are atomic orbitals of $H$ atoms. We can thus relax our RHF orbitals in such a way that they 
	finally form the above configuration at the dissociation limit. \\
	equation for parameter addition \\
	explaination for $\theta =0$ and 45 case. \\
	Let us consider the dissociated case of $H_2$ molecule and calculate its energy and wavefunction associated with it. \\
	
	\subsection{Discussion}
	The energy calculated above through the unrestricted wavefunction is matches our understanding of two dissociated H atoms. Although the unrestricted solution wavefunction $|\phi_A \phi_B> $ at the dissociation limit is not a pure spin 
	spin state and thus a wrong solution to the correct wavefucntion. Hartree Fock is a single determinant theory, and it finds a set of spin orbitals \Ki_A such that the single determinant produced is the best possible approximation to
	ground state of an N electron system. But as we shall discuss in the next section, the correct solution to $H_2$ dissociation at the dossociation limit is not a single determinant. \\
\section{Two Orbital two electron system}%
	\subsection{Spin Operator}
	\vv{s} =s_x \vv{i} + s_y \vv{j} + s_z \vv{j}
	Spin operator s is a vector quantity that describes the spin of an electron. Through Glen Fledrich experiemnet the following derivation for the x, y and z componcen of this operator are defined. 
	\vv{s_x}=
	\vv{s_x}=
	\vv{s_x}=
	Also through Glen Frenrich experiment it is clear that the spin operators do not commute with each other. This leads us to the following commutator relationships. \\
	
	
	\subsection{Spin Adapted Configurations}
\section{Configurational Interaction Solution}
	\subsection{Discussion}
	\subsection{Secular Equations and its solution}
\end{document}  
