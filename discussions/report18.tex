\documentclass[11pt]{article}   	% use "amsart" instead of "article" for AMSLaTeX format
\usepackage{geometry}                		% See geometry.pdf to learn the layout options. There are lots.
\geometry{letterpaper}                   		% ... or a4paper or a5paper or ... 
%\geometry{landscape}                		% Activate for rotated page geometry
%\usepackage[parfill]{parskip}    		% Activate to begin paragraphs with an empty line rather than an indent
\usepackage{graphicx}				% Use pdf, png, jpg, or eps§ with pdflatex; use eps in DVI mode
								% TeX will automatically convert eps --> pdf in pdflatex		
\usepackage{amssymb}
\usepackage{amsmath}
\usepackage{mathtools}
\usepackage{indentfirst}
%SetFonts

%SetFonts
\setlength{\parskip}{1em}
\begin{document}
\title{$H_2$ Dissociation Discussions}
\author{Cheng Lab}
\date{June 13 2018}							% Activate to display a given date or no date
\maketitle
\section{Introduction to Dissociation problem}
	%write at the end
\section{Restricted Hartree Fock Solution}
	%define basis and form MOs
	A simple qualitative description of dissociation process can be made by minimal basis 
	$H_2$ model using 1s orbitals on each $H$ atom as our basis
	i.e $\phi_A^{1s}$ and $\phi_B^{1s}$ as atomic orbitals on hydrogen atom A and B.  
	We will write $\alpha$ and $\beta$ spin as $\phi$ and $\bar{\phi}$ respectively. \\
	Restricted Hartree Fock orbitals, determined by pure symmetry considerations are :\\ 
	\begin{equation}
	\begin{split}
	\phi_{\sigma}=\frac{1}{\sqrt{2(1+S_{AB})}}(\phi_A+\phi_B) \\	
	\phi_{\sigma^*}=\frac{1}{\sqrt{2(1-S_{AB})}}(\phi_A-\phi_B) \\
	\end{split}
	\end{equation}
	where $S_{AB}$ is the overlap of atomic orbitals $\phi_A$ and $\phi_B$. The ground state staler determinant can be written as :\\
	\begin{equation}
	\begin{split}
	|\phi_{\sigma}\bar{\phi}_{\sigma}>=\frac{1}{\sqrt{2}} \begin{vmatrix}\phi_{\sigma}(1) 
	& \bar{\phi}_{\sigma}(1) \\  \phi_{\sigma}(2) & \bar{\phi}_{\sigma}(2) \\ \end{vmatrix}  \\
	\end{split}
	\end{equation}
	%edit the language
	
	The energy of this determinant based can be written by inspection to be $2h_{\sigma \sigma} + J_{\sigma \sigma}$ (one coulombic interaction 
	and no exchange possible between electrons of opposite spin) where h is the core 1 electron energy (kinetic and nuclear potential energy)
	and J is the coulombic 2 electron integral. Let us write our ground state determinant in terms of atomic orbitals.\\
	\begin{equation}
	\begin{split}
		|\phi_{\sigma}\bar{\phi}_{\sigma}> &=\frac{1}{\sqrt{2}}\big[ \phi_{\sigma}(1)\bar{\phi}_{\sigma}(2)
				-\phi_{\sigma}(2)\bar{\phi}_{\sigma}(1)\big]\\
			&=\frac{1}{2\sqrt{2}(1+S_{AB})}\big[(\phi_A(1)+\phi_B(1))(\bar{\phi}_A(2)+\bar{\phi}
			_B(2))-(\phi_A(2)+\phi_B(2))(\bar{\phi}_A(1)+\bar{\phi}_B(1))\big]\\
	\end{split}
	\end{equation}
	Expanding and rearranging, we get :\\
	\begin{equation}
	\begin{split}
	|\phi_{\sigma}\bar{\phi}_{\sigma}>=\frac{1}{2\sqrt{2}(1+S_{AB})}(|\phi_A \bar{\phi}_A>+|\phi_A \bar{\phi}_B>+|\phi_B \bar{\phi}_A>+|\phi_B \bar{\phi}_B>)\\
	\end{split}
	\end{equation}
	
	Note that $|\phi_A \bar{\phi}_A>$ and $|\phi_B \bar{\phi}_B>$ in the above configurations of Restricted Hartree Fock (RHF) ground 
	state determinant show ionic bond character. 
	These determinants are the reason of problematic behavior at dissociation of $H_2$ as we shall see in the later sections. 
	Evaluating total energy of this configuration, each electron has one core part $h_{\sigma \sigma}$ and 
	the two electrons have coulombic repulsion part $J_{\sigma \sigma}$, we get :\\
	\begin{equation}
		<\phi_{\sigma}\bar{\phi}_{\sigma}|H|\phi_{\sigma}\bar{\phi}_{\sigma}>
		=2 h_{\sigma \sigma}+J_{\sigma \sigma}\\
	\end{equation}
	
	We will analyse the two parts of energy one by one to see their behavior at the dissociation limit. \\
	\begin{equation}
	\begin{split}
		2 h_{\sigma \sigma}&=2 (\phi_{\sigma}\bar{\phi}_{\sigma}|h|
		\phi_{\sigma}\bar{\phi}_{\sigma})\\
		&=\frac{2}{2(1+S_{AB})}((\phi_A+\phi_B)(\bar{\phi}_A+\bar{\phi}_B)|h|(\phi_A+\phi_B)(\bar{\phi}_A+\bar{\phi}_B))\\
		&=\frac{1}{1+S_{AB}}(h_{AA}+h_{AB}+h_{BA}+h_{BB})\\
	\end{split}
	\end{equation}
	as $R\rightarrow \infty$ $S_{AB}$ and $h_{AB}\rightarrow 0$, we get :
	\begin{equation}
		2h_{\sigma \sigma}=h_{AA}+h_{BB}
	\end{equation}
	
	Similarly, using eq(4) and expanding $J_{\sigma \sigma}$ in atomic orbitals at dissociation limit, we get :
	\begin{equation}
	\begin{split}
		J_{\sigma \sigma} &=(\phi_{\sigma} \phi_{\sigma}|\phi_{\sigma} \phi_{\sigma})\\
			&= lim_{r\rightarrow\infty}\big( (\phi_A\phi_A|+(\phi_A\phi_B|+(\phi_B\phi_A|
			+(\phi_B\phi_B|\big)\big|\\&    \big(|\phi_A\phi_A)+|\phi_A\phi_B)+|\phi_B\phi_A)+
			|\phi_B\phi_B)\big)\\
			&=(\phi_A\phi_A|\phi_A\phi_A)+(\phi_B\phi_B|\phi_B\phi_B)\\
			&\neq 0
	\end{split}
	\end{equation}
	\subsection{Discussion of results}
	%discuss the source of the problem
	From eq(7) and eq(8), we infer that the RHF energy of dissociated $H_2$ molecule is $h_{AA}+h_{BB}+
	J_{AA}+J_{BB}$. Dissociated $H_2$ molecule essentially means two $ H $atoms at $\infty$ distance so 
	they cannot interact. The energy of this system should be twice the energy of $H$ atom. 
	Thus our RHF calculation is an overestimation of the exact energy.  \\
	
	If we look at eq(4) we find that there are four determinants contributing to the molecular configuration, two of which are ionic in nature (1st and 4th). 
	These determinants should have no contribution as the two orbitals $\phi_A$ and $\phi_B$ are spatially separated.
	This problem is due to the fact that RHF orbitals have the same spacial functions for $\alpha$ and $\beta$ spin. 
	Thus we need additional degree of freedom in our 
	orbital wavefunction to include different spacial wavefunctions for spin states. This results in an Unrestricted 
	Hartree Fock(UHF) method . \\
\section{Unrestricted Hartree Fock Solution}
	Restricted set of orbitals were generated purely by symmetry. Unrestricted molecular orbitals have different 
	spacial functions for $\alpha$ and $\beta$ spin of the same orbital. A formulation to incorporate additional degree of freedom in symmetry 
	determined RHF molecular orbitals is to form UHF orbitals as a linear combination of RHF orbitals. \\
	%We know that in the dissociated state $H_2$ minimal basis configuration will be $|\phi_A \bar{\phi}_B>$ where $\phi_a$ and $\phi_B$ are atomic orbitals of $H$ atoms. We can thus relax our RHF orbitals in such a way that they 
	%finally form the above configuration at the dissociation limit. \\
	%equation for parameter addition \\
	\begin{equation}
	\begin{split}
		\phi_{\sigma}^{\alpha}=cos\theta \phi_{\sigma}+sin\theta \phi_{\sigma^*}\\
		\phi_{\sigma}^{\beta}=cos\theta \phi_{\sigma}-sin\theta \phi_{\sigma^*}\\
	\end{split}
	\end{equation}
	\begin{equation}
	\begin{split}
		\phi_{\sigma^*}^{\alpha}=cos\theta \phi_{\sigma^*}+sin\theta \phi_{\sigma}\\
		\phi_{\sigma^*}^{\beta}=cos\theta \phi_{\sigma^*}-sin\theta \phi_{\sigma}\\
	\end{split}
	\end{equation}
	
	The added degree of freedom here is in $\theta$ which can be varied from $0^o$ to $45^o$. For our purposes it is sufficient to consider the 
	values at $0^o$ and $45^o$. At $0^o$ :\\
	\begin{equation}
		\begin{split}
		\phi_{\sigma}^{\alpha}&=\phi_{\sigma}\\
		\phi_{\sigma}^{\beta}&=\phi_{\sigma}\\
		\end{split}
	\end{equation}
	
	These are RHF ground state spin orbitals as discussed earlier. At $\theta=45^o$ we have :\\
	\begin{equation}
	\begin{split}
		\phi_{\sigma}^{\alpha}&=\frac{1}{\sqrt{2}}(\phi_{\sigma}+\phi_{\sigma^*})\\
		\phi_{\sigma}^{\beta}&=\frac{1}{\sqrt{2}}(\phi_{\sigma}-\phi_{\sigma^*})\\
	\end{split}
	\end{equation}
	
	In $\theta=45^o$, taking the dissociation limit case we have 
	$\phi_{\sigma}^{\alpha}=\phi_A$ and $\phi_{\sigma}^{\beta}=\phi_B$ which are unrestricted set of 
	molecular orbitals.
	With the new set of orbitals our determinant becomes $|\phi_{\sigma}^{\alpha}\phi_{\sigma}^{\beta}>
	=|\phi_A\bar{\phi}_B>$.\\ 
	
	Let us calculate the energy of the ground state determinant at the dissociation limit with our new UHF orbitals. \\
	\begin{equation}
		\begin{split}
			lim_{r\rightarrow \infty}<\Psi_o|H|\Psi_o>&=<\phi_{\sigma}^{\alpha}
			\phi_{\sigma}^{\beta}|H|\phi_{\sigma}^{\alpha}\phi_{\sigma}^{\beta}>\\
				&=<\phi_A\phi_B|H|\phi_A\phi_B>\\
				&=h_{AA}+h_{BB}\\
		\end{split}
	\end{equation}
	
	As shown here unlike RHF, UHF orbitals show correct behavior of energy at the dissociation. 
	limit of $H_2$ molecule.\\
	\subsection{Discussion}
	We see that the energy by UHF wavefunction $lim_{r\rightarrow\infty}|\Psi_o>=|\phi_{\sigma}^{\alpha}\phi_{\sigma}^{\beta}>$ goes to
	correct limit but the wavefunction is incorrect. The UHF wavefunction at the dissociation limit
	 becomes $|\phi_A\bar{\phi}_B>$ in terms of atomic orbitals. This is not a pure spin state in the case where electrons occupy
	 different spacial orbitals as in the UHF orbitals. We will discuss spin adaptation in detail in section 3.
	For reference, the correct wavefunction is :\\
	\begin{equation}
	\begin{split}
		Lim_{r\rightarrow\infty}|\Psi_o>=\frac{1}{\sqrt{2}}(|\phi_A\bar{\phi}_B>+|\bar{\phi}_A\phi_B>)
	\end{split}
	\end{equation}

	In terms of UHF orbitals, this correct  wavefunction is a multi-determinant wavefunction :\\
	\begin{equation}
	\begin{split}
		Lim_{r\rightarrow\infty}|\Psi_o>=\frac{1}{\sqrt{2}}(|\phi_{\sigma}^{\alpha}\bar
		{\phi}_{\sigma}^{\beta}>+|\bar{\phi}_{\sigma}^{\beta}\phi_{\sigma}^{\beta}>)
	\end{split}
	\end{equation}
	
\subsection{Spin Operator}
	Spin Operator is a vector quantity defined as follows :
	\begin{equation}
		\vec{s}=s_x \vec{i}+s_y\vec{j} + s_z\vec{k}\\
	\end{equation}
	
	These spin operator components do not commute with each other and satisfy the following relations :
	\begin{equation}
	\begin{split}
		[s_x,s_y]=is_z,   
		[s_y,s_z]=is_x,   
		[s_z,s_x]=is_y
	\end{split}
	\end{equation}
	
	We can derive a set of states of the spin of a particle as eigenfunctions of $s^2$ or one of the components of $\vec{s}$ conventionally taken to be $s_z$. \\
	\begin{equation}
	\begin{split}	
		s^2|\phi>&=s(s+1)|\phi>\\
		s_z|\phi>&=m_s|\phi>
	\end{split}
	\end{equation}
	where $s$ is the total spin of the particle and $m_s$ is the quantum number for the z component. 
	Since electron is a spin half system, where $m_s=\frac{1}{2} , -\frac{1}{2}$. 
	For our convenience, we can define the two states in the matrix representation as follows :\\
	\begin{equation}
		|\alpha>=\Big(\begin{matrix} 1\\ 0\end{matrix}\Big), 
		|\beta>=\Big(\begin{matrix} 0\\ 1\end{matrix}\Big)\\
	\end{equation}
	
	This leads us to a more convenient definition of ladder operators:
	\begin{equation}
	\begin{split}
	s_+=\bigg(\begin{matrix} 0 & 1\\ 0 & 0\end{matrix}\bigg)\\
	s_-=\bigg(\begin{matrix} 0 & 0\\ 1 & 0\end{matrix}\bigg)\\
	\end{split}
	\end{equation}
	where the action of these ladder operators can be seen as taking the spin state up 
	 through $s_+$ and down through $s_-$. They are 
	defined in terms for x and y components of $\vec{s}$ as follows :
	\begin{equation}
	\begin{split}
		s_+&=s_x+is_y\\
		s_-&=s_x-is_y\\
	\end{split}
	\end{equation}
	
	Using eq(17), (19) and (20) we can write the x, y and z components of our spin operator in the matrix 
	representation as $s_k=\frac{1}{2} \sigma_k$. Here k=x, y and z and $\sigma_k$ is defined as follows:\\
	\begin{equation}
		\begin{split}
		\sigma_1=\bigg(\begin{matrix}0 & 1\\ 1 & 0\end{matrix}\bigg),
		\sigma_2=\bigg(\begin{matrix}0 & -i\\ i & 0\end{matrix}\bigg), 
		\sigma_3=\bigg(\begin{matrix}1 & 0\\ 0 & -1\end{matrix}\bigg)\\ 
		\end{split}
	\end{equation}
	here 1, 2 and 3 refer to x, y and z components. This matrix formalism for spin $\frac{1}{2}$ systems 
	was introduced by W. Pauli in 1926 and the matrices $\sigma_k$ are called Pauli matrices. 
	We can also derive an expression for $s^2$ in terms of components
	of $\vec{s}$ using eq(15) and (16) :\\
	\begin{equation}
	\begin{split}
		s^2=s_+s_--s_z+s_z^2\\
		s^2=s_-s_++s_z+s_z^2\\
	\end{split}
	\end{equation}
	
	For many electron systems, the total spin angular momentum is the sum of spin vectors for each 
	electron.\\
	\begin{equation}
	\begin{split}
		\vec{S}=\sum_{i=1}^N \vec{s}\big(i\big)\\
	\end{split}
	\end{equation}	
	
	Thus the total squared spin operator is :\\
	\begin{equation}
	\begin{split}
		S^2 &=\vec{S} .\vec{S}=\sum_{i=1}^N \sum_{j=1}^N \vec{s}\big(i\big)\vec{s}\big(j\big)\\
			&=S_+S_--S_z+S_z^2\\
			&=S_-S_++S_z+S_z^2
	\end{split}
	\end{equation}
	
	Since there is no spin part in our non-relativistic $H$, $S^2$ and $S_z$ commutes with $H$ 
	i.e $[H,S^2]=0$	and $[H,S_z]=0$. Thus the exact eigenfunctions of Hamiltonian (exact wavefunctions) 
	are also eigenfunctions of $S^2$ and $S_z$ operators.\\ 
	
	Approximate solutions of $H$ are not necessarily pure spin states. However we would like to convert
	them to spin adapted configurations to form eigenfunctions of $S^2$. We will describe the process 
	through an example in the following sections. \\
	

	\subsection{Two Orbital two electron system}
	To understand spin adapted configurations, let us consider an example of a simple case of two electrons in two 
	orbitals $\phi_1 \phi_2$. Two electrons can be arranged in two orbitals in the following 6 ways: \\
	\begin{equation}
		|\phi_1\bar{\phi}_1>, |\phi_2\bar{\phi}_2>, |\phi_1\bar{\phi}_2>, |\bar{\phi}_1\phi_2>, |\phi_1\phi_2>, |\bar{\phi}_2\bar{\phi}_1>\\ 
	\end{equation}
	
	If we calculate the $S$ value for the above configurations, we can see that $|\phi_1\bar{\phi}_2>
	and  |\bar{\phi}_1\phi_2>$ are not pure spin states. \\
	\begin{equation}
	\begin{split}
	S^2|\phi_1\bar{\phi}_2>&=S^2\big[\phi_1(1)\phi_2(2)\alpha(1) \beta(2)-\phi_1(2)\phi_2(1)\alpha(2) \beta(1)\big]\\
		&=\big[\phi_1(1)\phi_2(2)-\phi_1(2)\phi_2(1)\big](\alpha(1) \beta(2)+\alpha(2)\beta(1))
	\end{split}
	\end{equation}
	
	We can make them eigenfunctions of $S^2$ operator by taking a linear combination of appropriate terms
	\\
	\begin{equation}
	\begin{split}
		|\prescript{1}{}{\Psi}>=\frac{1}{\sqrt{2}}(|\phi_1\bar{\phi}_2>+|\phi_1\bar{\phi}_2>)\\	
		|\prescript{3}{}{\Psi}>=\frac{1}{\sqrt{2}}(|\phi_1\bar{\phi}_2>-|\phi_1\bar{\phi}_2>)\\
	\end{split}
	\end{equation}
	
	As discussed here, we now have three triplet and three singlet spin adapted states 
	from two electrons in two orbitals model. Writing in terms of $H_2$ case that we discussed in 
	section (2) and (3) we have the following singlet states :\\
	\begin{equation}
	\begin{split}
		|\prescript{1}{}{\Psi_1}>&=|\phi_{\sigma}\bar{\phi}_{\sigma}>\\
		|\prescript{1}{}{\Psi_2}>&=|\phi_{\sigma^*}\bar{\phi}_{\sigma^*}>\\
		|\prescript{1}{}{\Psi_3}>&=\frac{1}{\sqrt{2}}(|\phi_{\sigma}\bar{\phi}_{\sigma^*}>+
		|\phi_{\sigma}\bar{\phi}_{\sigma^*}>)\\	
	\end{split}
	\end{equation}
	
	Similarly the triplet states are :\\
	\begin{equation}
	\begin{split}
		|\prescript{3}{}{\Psi_4}>&=|\phi_{\sigma}\phi_{\sigma^*}>\\
		|\prescript{3}{}{\Psi_5}>&=|\bar{\phi}_{\sigma^*}\bar{\phi}_{\sigma}>\\
		|\prescript{3}{}{\Psi_6}>&=\frac{1}{\sqrt{2}}(|\phi_{\sigma}\bar{\phi}_{\sigma^*}>-
		|\phi_{\sigma}\bar{\phi}_{\sigma^*}>)\\
	\end{split}
	\end{equation}
\section{Configurational Interaction Solution}
	One way to look at our HF solutions of $H_2$ dissociation is to follow the two molecular orbitals produced $\phi_{\sigma}$ and $\phi_{\sigma^*}$. 
	The energy of configuration $|\phi_{\sigma} \bar{\phi}_{\sigma}>$ is $2h_{\sigma} +J_{\sigma \sigma}$ and of configuration 
	$|\phi_{\sigma^*} \bar{\phi}_{\sigma^*}>$ is $2h_{\sigma^*} +J_{\sigma^* \sigma^*}$. If we expand the molecular orbitals and 
	write the energy in terms of atomic orbitals, we get:\\
	\begin{equation}
	\begin{split}
	2h_{\sigma \sigma} +J_{\sigma \sigma}=h_{AA}+h_{AB}+h_{BA}+h_{BB}+J_{AA}+J_{BB}\\
	2h_{\sigma^* \sigma^*} +J_{\sigma \sigma}=h_{AA}-h_{AB}-h_{BA}+h_{BB}-J_{AA}+J_{BB}\\
	\end{split}
	\end{equation}
	At the dissociation limit, $h_{AB}=h_{BA}=0$ as $H_A$ and $H_B$ have no interaction. 
	Thus the two states become degenerate. In HF we are calculating the energy of one of these 
	states i.e $\phi_{\sigma}$. The total energy is further reduced by what is known as the non-
	dynamic correlation or the mixing of the two degenrate states. 
	One can also look at this degeneracy arising in eq() in UHF case. 
	Since the two states have the same energy, it should be appropriate to chose a linear 
	combination of the two states as our ground state wavefunction and use the configuration 
	interaction (CI) method. The ground state CI wavefunction will be :\\
	\begin{equation}
		|\Phi_{CI}>=c_1|\prescript{1}{}{\Phi_{\sigma^2}}>+c_2|\prescript{1}{}{\phi_{\sigma^*2}}>
	\end{equation}
	calculating the energy we have :\\
	\begin{equation}
	\begin{split}
		E&=<\Phi_{CI}|H|\Phi_{CI}>\\
		 &=\big( \begin{matrix} c_1^* & c_2^* \end{matrix} \big)\bigg(\begin{matrix} H_{11} & H_{12} \\ H_{21} & H_{22} \end{matrix}\bigg)
			 \bigg( \begin{matrix} c_1\\c_2\end{matrix}\bigg)\\
		 &=c_1^*c_1H_{11}+c_1^*c_2h_{12}+c_2^*c_1H_{21}+c_2^*c_2H_{22}\\
	\end{split}
	\end{equation}
	where $H$ matrix elements are defined as follows :\\
	\begin{equation}
	\begin{split}
		H_{11}&=<\phi_{\sigma}\phi_{\sigma}|H|\phi_{\sigma}\phi_{\sigma}>=2 h_{\sigma\sigma}+J_{\sigma\sigma}\\
		H_{12}&=<\phi_{\sigma}\phi_{\sigma}|H|\phi_{\sigma^*}\phi_{\sigma^*}>=(\phi_{\sigma}
		\phi_{\sigma}|\phi_{\sigma^*}\phi_{\sigma^*})\\
		H_{21}&=<\phi_{\sigma^*}\phi_{\sigma^*}|H|\phi_{\sigma}\phi_{\sigma}>=(\phi_{\sigma^*}
		\phi_{\sigma^*}|\phi_{\sigma}\phi_{\sigma})\\
		H_{22}&=<\phi_{\sigma^*}\phi_{\sigma^*}|H|\phi_{\sigma^*}\phi_{\sigma^*}>=2 h_{\sigma^*\sigma^*}+J_{\sigma^*\sigma^*}\\
	\end{split}
	\end{equation}


	Normalisation condition of $\Phi_{CI}$ leads to :\\
	\begin{equation}
	\begin{split}
		<\Phi_{CI}|\Phi_{CI}>&=1\\
		c_1^*c_1<\Psi_{\sigma^2}|\Psi_{\sigma^2}>+c_2^*c_2<\Psi_{\sigma^{*2}}|\Psi_{\sigma^{*2}}>&=1\\
	\end{split}	
	\end{equation}
	
	To solve for the minimum in energy eq(33) based on the above constraint eq(35), we form the Lagrange function and
	 find its minima.  
	\begin{equation}
	\begin{split}
	L&=\sum_{ij}c_i^*c_j<\phi_i|H_{ij}|\phi_j>-E(\sum_{i,j}c_i^*c_j<\phi_i|\phi_j>-1)\\
	\delta L&=\sum_i\delta c_i^*\big[\sum_j H_{ij}c_j-ES_{ij}c_j\big]+c.c\\
	\end{split}
	\end{equation}
	Here S is the overlap matrix of our basis. The $\delta L$ equations are simuntanious 
	equations in $c_1$ and $c_2$, and are called secular equations. Since the inner part of the 
	eq(36) has to go to 0 for secular equations to vanish, thus we form
	the secular determinant:\\
	\begin{equation}
	\begin{split}
		Hc=ScE
	\end{split}
	\end{equation}
	here, $c$ is a matrix of eigenvectors and $E$ is a diagonal matrix of eigenvalues of $H$.\\

	Our basis is orthogonal thus we can reduce this equation to $Hc=cE$. Solving this equation in the 
	matrix form can be done by diagoanalisation of $H$ as :
	\begin{equation}
	\begin{split}
		Hc&=cE\\
		H&=cEc^{-1}\\
	\end{split}
	\end{equation}
	diagonalising $H$ matrix therefore gives c as eigenvector matrix and energy in the diagonal 
	eigenvalue matrix. Solving this equation in our example calculation
	 we get our minimum energy at $c_1=\frac{1}{\sqrt{2}}$ and $c_2=-\frac{1}{\sqrt{2}}$. Our
	 ground state wavefunction is :\\
	\begin{equation}
	\begin{split}
		\Phi_{CI}=\frac{1}{\sqrt{2}}\big(\Psi_{\sigma^2}-\Psi_{\sigma^*2}\big)
	\end{split}
	\end{equation}
	Unlike UHF and RHF, $\Phi_{CI}$ has equal contribution from both the degenerate state. 
	Also, $E_{CI}=h_{AA}+h_{BB}$ which is the correct dissociation limit energy.\\ 
	\subsection{Discussion}	
	\subsection{Secular Equations and its solution}
	Another way to derive the same secular determinant is th
\end{document}  
