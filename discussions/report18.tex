\documentclass[11pt]{article}   	% use "amsart" instead of "article" for AMSLaTeX format
\usepackage{geometry}                		% See geometry.pdf to learn the layout options. There are lots.
\geometry{letterpaper}                   		% ... or a4paper or a5paper or ... 
%\geometry{landscape}                		% Activate for rotated page geometry
%\usepackage[parfill]{parskip}    		% Activate to begin paragraphs with an empty line rather than an indent
\usepackage{graphicx}				% Use pdf, png, jpg, or eps§ with pdflatex; use eps in DVI mode
								% TeX will automatically convert eps --> pdf in pdflatex		
\usepackage{amssymb}
\usepackage{amsmath}
\usepackage{indentfirst}
%SetFonts

%SetFonts
\setlength{\parskip}{1em}




\begin{document}
\title{$H_2$ Dissociation Discussions}
\author{Cheng Lab}
\date{June 13 2018}							% Activate to display a given date or no date
\maketitle
\section{Introduction to Dissociation problem}
	%write at the end
\section{Restricted Hartree Fock Solution}
	%define basis and form MOs
	For minimum basis $H_2$ we have two 2s orbitals, one on each H atom as our basis.
	i.e \phi_A and \psi_B as atomic orbitals on hydrogen atom A and B. Let $\bar{\phi_A}$ and $\bar{\phi_B}$
	denote the \beta spin. \\
	Restricted Hartree Fock orbitals will be determined by taking a linear combination of atomic orbitals and minimising the energy through those orbitals. 

	$\phi_{\sigma}=\frac{1}{\sqrt{1+S_{12}}}(\phi_A+\phi_B)$\\	
	$\phi_{\sigma}=\frac{1}{\sqrt{1+S_{12}}}(\phi_A+\phi_B)$\\
	The determinant then becomes \\

	$|phi_{\sigma}\bar{\phi_{sigma}}>=\frac{1}{sqrt{2}}matrix$
	$|phi_{\sigma}\bar{\phi_{sigma}}>=\frac{1}{sqrt{2}}matrix$
		
	%solve the HF equation to get the energy of the present system
	%discuss what the energy should be and prove that J is not zero
	\subsection{Discussion of results}
	%discuss the source of the problem
\section{Unrestricted Hartree Fock Solution}
	\subsection{Discussion}
\section{Two Orbital two electron system}%
	\subsection{Spin Operator}
	\subsection{Spin Adapted Configurations}
\section{Configurational Interaction Solution}
	\subsection{Discussion}
	\subsection{Secular Equations and its solution}
\end{document}  
