\documentclass[11pt, oneside]{article}   	% use "amsart" instead of "article" for AMSLaTeX format
\usepackage{geometry}                		% See geometry.pdf to learn the layout options. There are lots.
\geometry{letterpaper}                   		% ... or a4paper or a5paper or ... 
%\geometry{landscape}                		% Activate for rotated page geometry
%\usepackage[parfill]{parskip}    		% Activate to begin paragraphs with an empty line rather than an indent
\usepackage{graphicx}				% Use pdf, png, jpg, or eps§ with pdflatex; use eps in DVI mode
								% TeX will automatically convert eps --> pdf in pdflatex		
\usepackage{amssymb}
\usepackage{amsmath}

%SetFonts

%SetFonts


\title{Unitary Coupled Cluster Theory }
\author{Cheng Lab}
%\date{}							% Activate to display a given date or no date

\begin{document}

\maketitle

\section{Derivation of Polarisation Propogator $G_{pr,rs}$ in  the new unitary transformed excitation/deexcitation scheme}
The Polarisation Propogator in superoperator representation is given by :

\begin{equation*}\label{eq:propogator}
  \begin{aligned}
    G_{pq,rs}\big(\omega\big)&=\big \langle \Psi_{gr} |\big[ a_p a_q^\dagger , \big(\omega \hat I-\hat H \big)^{-1} a_r^\dagger a_s \big] | \Psi_{gr}\big \rangle \\
   & =\big \langle \Psi_{gr} |\big[ q_I , \big(\omega \hat I-\hat H \big)^{-1} q_J^\dagger \big] | \Psi_{gr}\big \rangle \\ 
  \end{aligned}
\end{equation*}
A Binary Product can be defined as follows :

\begin{equation*}\label{eq:binary product}
  \begin{aligned}
    (A|B)=\big \langle \Psi_{gr}|\big[A^\dagger,B]|\Psi\rangle
  \end{aligned}
\end{equation*}
Another definition used in this derivation is :

\begin{equation*}\label{eq:superoperator}
  \begin{aligned}
    (A|\hat O |B)\equiv (A|\hat O B)
  \end{aligned}
\end{equation*}

Polarisation Propogator can be written in terms of binary product notation as :
\begin{equation*}\label{eq:superoperator}
  \begin{aligned}
    G(\omega)=(q_i^\dagger|(\omega \hat I - \hat H)^{-1}q_j^\dagger )
  \end{aligned}
\end{equation*}
In our self consistant polarisation propogation theory, the unitary transformed excitation/deexcitation operators are defined as :

\begin{equation*}\label{eq:unitary transformed excitation}
  \begin{aligned}
    y_i^\dagger=\exp(\sigma) b_i^\dagger \exp(- \sigma)
  \end{aligned}
\end{equation*}
Using the resolution of identity in G :

\begin{equation*}\label{eq:resolution of identity}
  \begin{aligned}
    \hat I=\sum_I \{ |y_I^\dagger)(y_I^\dagger|-|y_I)(y_I|\}\\
    G(\omega)=(q_I|y_I^\dagger)(y_I^\dagger|(\omega \hat I - \hat H)^{-1}q_j^\dagger) - (q_I|y_I)(y_I|(\omega \hat I - \hat H)^{-1}q_j^\dagger)
  \end{aligned}
\end{equation*}
Taking the second half of the expression and expanding through tailors expansion :
\begin{equation*}\label{eq:resolution of identity}
  \begin{aligned}
    (y_I|(\omega \hat I - \hat H)^{-1}q_j^\dagger) &=\big \langle \Psi_{gr} |\big[ y_I , \big(\omega \hat I-\hat H \big)^{-1} q_j^\dagger \big] | \Psi_{gr}\big \rangle \\ 
     &=\big \langle \Psi_{gr} |\big[ y_I , \frac{1}{ \omega} \Bigg( \big( \hat I-\hat H \big)\Bigg)^{-1} q_j^\dagger \big] | \Psi_{gr}\big \rangle \\ 
     &=\big \langle \Psi_{gr} |\big[ y_I , \frac{1}{\omega} \Bigg( \big( \hat I+ \frac{1}{ \omega} \hat H + \frac{1}{\omega^2}\hat H^2 ...\big)\Bigg) q_j^\dagger \big] | \Psi_{gr}\big \rangle \\ 
     &=\frac{1}{\omega }\Bigg(\big \langle \Psi_{gr} |\big[ y_I , \big( \hat I q_j^\dagger \big] | \Psi_{gr}\big \rangle +\big \langle \Psi_{gr} |\big[ y_I , \frac{1}{\omega}\hat H  q_j^\dagger \big] | \Psi_{gr}\big \rangle  +\big \langle \Psi_{gr} |\big[ y_I ,\frac{1}{\omega^2} \hat H^2  q_j^\dagger \big] | \Psi_{gr}\big \rangle \Bigg)\\ 
     &=\frac{1}{\omega} \Bigg(\big( y_I | \big( \hat I q_j^\dagger \big) +\big( y_I | \frac{1}{\omega}\hat H  q_j^\dagger \big) +\big( y_I |\frac{1}{\omega^2} \hat H^2  q_j^\dagger \big)\\ 
     &=\frac{1}{\omega} \Bigg(\big( y_I | \big( \hat I| q_j^\dagger \big) +\big( y_I | \frac{1}{\omega}\hat H | q_j^\dagger \big) +\big( y_I |\frac{1}{\omega^2} \hat H^2|  q_j^\dagger \big)...\Bigg)\\ 
     &=\frac{1}{\omega} \big( y_I | \big( \hat I + \frac{1}{\omega}\hat H + \frac{1}{\omega^2} \hat H^2 ...|  q_j^\dagger \big)\\ 
     &=\frac{1}{\omega} \big( y_I | \big( \hat I - \frac{\hat H}{\omega}\big)^{-1}|  q_j^\dagger \big)\\ 
     &= \big( y_I | \big( \omega\hat I - \hat H\big)^{-1}|  q_j^\dagger \big)\\ 
  \end{aligned}
\end{equation*}
Using this result in the previously derived expression :

\begin{equation*}\label{eq:resolution of identity}
  \begin{aligned}
    \hat I&=\sum_I \{ |y_I^\dagger)(y_I^\dagger|-|y_I)(y_I|\}\\
    G(\omega)&=(q_I^\dagger|y_I^\dagger)(y_I^\dagger|(\omega \hat I - \hat H)^{-1}q_j^\dagger) - (q_I^\dagger|y_I)(y_I|(\omega \hat I - \hat H)^{-1}q_j^\dagger)\\
    &=(q_I^\dagger|y_I^\dagger)(y_I^\dagger|(\omega \hat I - \hat H)^{-1}|q_j^\dagger) - (q_I^\dagger|y_I)(y_I|(\omega \hat I - \hat H)^{-1}|q_j^\dagger)
  \end{aligned}
\end{equation*}
Again using resolution of Identity Operator : 
\begin{equation*}\label{eq:resolution of identity}
  \begin{aligned}
    G(\omega)&=(q_I^\dagger|y_I^\dagger)(y_I^\dagger|(\omega \hat I - \hat H)^{-1}|y_J^\dagger)(y_J^\dagger|q_j^\dagger) - (q_I^\dagger|y_I)(y_I|(\omega \hat I - \hat H)^{-1}|y_J^\dagger)(y_J^\dagger|q_j^\dagger)\\
 &- (q_I^\dagger|y_I^\dagger)(y_I^\dagger|(\omega \hat I - \hat H)^{-1}|y_J)(y_J|q_j^\dagger) + (q_I^\dagger|y_I)(y_I|(\omega \hat I - \hat H)^{-1}|y_J)(y_J|q_j^\dagger)
  \end{aligned}
\end{equation*}

\section{Resolution of Identity}
\subsection{Finding Identity for the notations}




\end{document}  
